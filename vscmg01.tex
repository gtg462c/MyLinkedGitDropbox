\documentclass[12pt]{article}
%\usepackage{lslide}
\usepackage{epsfig}
\newcommand{\beq}{\begin{equation}}
\newcommand{\eeq}{\end{equation}}
\newcommand{\beqa}{\begin{eqnarray}}
\newcommand{\eeqa}{\end{eqnarray}}
\newcommand{\bi}{\begin{itemize}}
\newcommand{\ei}{\end{itemize}}
%\newcommand{\IP}{\{}^P \[ I_{P/Cp}\]}
\pagenumbering{Roman}
\setlength{\textheight}{23cm}
\setlength{\textwidth}{16cm}
\setlength{\oddsidemargin}{0cm}
\setlength{\evensidemargin}{0cm}
\setlength{\topmargin}{0.5cm}
\setlength{\topskip}{0cm}
\setlength{\footskip}{1cm}

% ----------------------------------------------------------------------
\title{SPACECRAFT ATTITUDE CONTROL AND POWER TRACKING WITH SINGLE-GIMBALED
VSCMGs}
\author{ Hyungjoo Yoon \\ School of Aerospace Engineering\\ Georgia Institute of
Technology}
% ----------------------------------------------------------------------
\begin{document}
%%% ----------------------------------------------------------------------
\maketitle

\section{Introduction}
This report is a summary of the recent research on
spacecraft attitude control and power tracking with
single-gimbaled variable-speed control moment gyroscopes(VSCMGs).
I mainly did this research on the bases of some previous
literatures of Schaub {\it et.al.}\cite{schaub1}\cite{schaub2}, Ford {\it et.al}\cite{ford1}\cite{ford2}, and Tsiotras {\it et.al.}\cite{tsiotras1}.

\section{Equation of Motion}

\begin{figure}[h]
\centering
\epsfig{file=vscmg01.eps,width=10cm}
\caption{Spacecraft Body with a VSCMG}
\label{fig01}
\end{figure}

Figure \ref{fig01} is a spacecraft body with a VSCMG. A body(B)
consists of a platform(P) and multiple CMGs(C). Each of CMGs is
composed of a gimbal structure(G) and a momentum wheel(W).
When $\gamma$ is a column vector of each gimbal angles and $\Omega$ is
of angular speed of each momentum wheels, the angular momentum of whole S/C body including the platform and CMGs
can be expressed in B-frame as
\beq
h = J \omega + A_g [I_{cg}]\dot{\gamma} +
A_s [I_{ws}]\Omega
\eeq
where,
\beq
J(\gamma) = {}^B [I] + A_s [I_{cs}]A_s^T + A_t [I_{ct}]A_t^T + A_g [I_{cg}]A_g^T
\eeq
and
${}^B[I]$ is a matrix of inertia of a platform and point-mass
CMGs. $A_s$ is the matrix defined as
\beq
A_s = \left[ \hat{e}_{s1},\cdots ,\hat{e}_{sN} \right]
\eeq
and $A_t$ and $A_g$ are also defined in similar way.

In addition, $[I_{cs}]$,$[I_{ct}]$ and $[I_{cg}]$ are diagonal
matrices whose elements are moments of inertia of N-CMGs(gimbal structure and wheel) along
each axis, while $[I_{ws}]$ is a diagonal matrix whose elements
are the spin-axis moments of inertia of wheels only.

If $h_c$ is defined as $A_g [I_{cg}]\dot{\gamma} +
A_s [I_{ws}]\Omega$, then the time derivative of $h$ with respect to
B-frame is
\beqa
\dot{h} &=& \dot{J}\omega + J \dot{\omega} + \dot{h}_c \label{eq01} \\
&=& h \times \omega + g_e \label{eq02}
\eeqa
where, $g_e$ is a external torque.

Since,
\[ \dot{A}_s = A_t [diag(\dot{\gamma})], \dot{A}_t = - A_s
[diag(\dot{\gamma})]\] the time derivatives of $J$ and $h_c$ in
Eq.(\ref{eq01}) are \beqa \dot{h}_c &=& A_g [I_{cg}]\ddot{\gamma}
+ \dot{A}_s [I_{ws}]\Omega + A_s [I_{ws}]\dot{\Omega} \\ &=&A_g
[I_{cg}]\ddot{\gamma} + A_t [diag(\dot{\gamma})] [I_{ws}]\Omega +
A_s [I_{ws}]\dot{\Omega} \eeqa and \cite{ford1} \beq \dot{J} = A_t
[diag(\dot{\gamma})]([I_{cs}]-[I_{ct}])A_s^T + A_s
[diag(\dot{\gamma})]([I_{cs}]-[I_{ct}])A_t^T \eeq

\section{Lyapunov Feedback Control Law}

Suppose a Lyapunov function is defined as \cite{tsiotras1}
\beq V
= \frac{1}{2} (\omega - \omega_r)^T J (\omega - \omega_r) + 2 k_0
\ln{ ( 1 + \sigma^T_e \sigma_e ) }\eeq
where, $k_0$ is a positive
scalar constant, $\omega_r$ is a reference (target) body rate
written in B-frame, and $\sigma_e$ is the error Modified Rodrigues
Parameters(MRP) vector between the reference frame and the
B-frame; i.e., \beq [R^B_R (\sigma_e )] =[R^B_N (\sigma)] [R^R_N
(\sigma_r )]^T \eeq The kinematics equation of the error MRP is
\beq \dot{\sigma_e} = G(\sigma_e ) (\omega -\omega_r ) \eeq where,
\beq G(\sigma_e ) = \frac{1}{2} \left( I_3 + [\sigma_e \times ] +
\sigma_e \sigma_e^T - \frac{1+\sigma_e^T \sigma_e}{2} I_3 \right)
\eeq

Them, the time derivative of this positive definite function $V$
is
\beqa
\dot{V} &=& \frac{1}{2} (\omega - \omega_r)^T \dot{J} (\omega - \omega_r)
+ (\omega - \omega_r)^T J (\dot{\omega} - \dot{\omega}_r)
+ 2 k_o \frac{2 \sigma_e^T \dot{\sigma}_e}{1 + \sigma^T_e
\sigma_e}\\
&=& -(\omega-\omega_r )^T \left\{ -\frac{1}{2} \dot{J} (\omega-\omega_r
)- J(\dot{\omega} - \dot{\omega}_r) - k_0 \sigma_e \right\}
\eeqa
For Lyapunov stability,
\beq
-\frac{1}{2} \dot{J} (\omega-\omega_r
)- J(\dot{\omega} - \dot{\omega}_r) - k_0 \sigma_e = k_1
(\omega-\omega_r )
\eeq
where, $k_1$ is a 3$\times$3 positive definite matrix.
From Eq.(\ref{eq01}) and (\ref{eq02}), the sufficient condition for the Lyapunov stability is
\beq
\dot{h}_c + \frac{1}{2} \dot{J} (\omega+\omega_r ) = k_1 (\omega-\omega_r )
+ k_0 \sigma_e - J \dot{\omega}_r + h \times \omega + g_e
\eeq
If the right side of this equation is defined as a required control torque $L_r$, this
condition can be expressed in a very compact form as follow;
\beq
B \ddot{\gamma} + C\dot{\gamma} + D\dot{\Omega} = L_r \label{eq03}
\eeq
where \cite{ford2},
\beqa
B &=& A_g [ I_{cq} ] \\
D &=& A_s [ I_{ws} ] \\
C &=& A_t [ I_{ws} ][diag(\Omega)] \\ &+& \frac{1}{2}
[(\hat{e}_{s1}\hat{e}_{t1}^T +
\hat{e}_{t1}\hat{e}_{s1}^T)(\omega+\omega_r ) , \cdots ,(\hat{e}_{sN}\hat{e}_{tN}^T +
\hat{e}_{tN}\hat{e}_{sN}^T)(\omega+\omega_r )]
([I_{cs}]-[I_{ct}])
\eeqa

\section{Steering Law for Attitude and Power Tracking}
\subsection{Velocity based steering law}
For the advantage of the torque amplification effect of CMGs,
most of the required control torque vector $L_r$ should be
produced by the $D\dot{\gamma}$ term. So, if gimbal acceleration
term $B \ddot{\gamma}$ is ignored in this level, the stability
condition becomes
\beq
Q\dot{\eta} = L_r \label{eq04}
\eeq
where,
\beq
Q = [ C \vdots D ] , \; \; \dot{\eta} = \left\{ \begin{array}{c}
\dot{\gamma} \\ \dot{\Omega} \end{array} \right\}
\eeq
Eq.(\ref{eq04}) is underdetermined equation, so there exist
infinitely many solutions of $\dot{\eta}$.
One of the solutions is the minimum norm solution; i.e.,
\beq
\dot{\eta} = Q^T ( Q Q^T )^{-1} L_r
\eeq
However, because ideally the VSCMGs are to act like classical CMGs
away from single-gimbal CMG singular configurations, a weighted
pseudo inverse can be used instead. Then, the desired $\dot{\eta}$
is \cite{schaub1}
\beq
\dot{\eta} = WQ^T(QWQ^T)^{-1} L_r
\eeq

\subsection{Power tracking}

The general solution of $\dot{\eta}$ in Eq.(\ref{eq04}) is given by
\beq
\dot{\eta} = WQ^T(QWQ^T)^{-1} L_r + \dot{\eta}_n
\eeq
where $\dot{\eta}_n$ is on the null space of $Q$; i.e.,
\beq
Q \dot{\eta}_n = 0
\eeq

The total kinetic energy stored in the momentum wheels is
\beq
T = \frac{1}{2} \Omega^T [I_{ws}] \Omega
\eeq
So the power is given by
\beqa
P &=& \frac{dT}{dt} = \Omega^T [I_{ws} \;] \dot{\Omega} \\
&=& \left[\; 0 \; \vdots \; \Omega^T [I_{ws}] \;\right] \left\{ \begin{array}{c}
\dot{\gamma} \\ \dot{\Omega} \end{array} \right\} \\
&=& \left[\; 0 \; \vdots \; \Omega^T [I_{ws}] \;\right]WQ^T(QWQ^T)^{-1} L_r
+\left[\; 0 \; \vdots \; \Omega^T [I_{ws}] \;\right] \left\{ \begin{array}{c}
\dot{\gamma}_n \\ \dot{\Omega}_n \end{array} \right\}
\eeqa
Therefore,
\beq
\left[\; 0 \; \vdots \; \Omega^T [I_{ws}] \;\right] \left\{ \begin{array}{c}
\dot{\gamma}_n \\ \dot{\Omega}_n \end{array} \right\} = P -\left[\; 0 \; \vdots \; \Omega^T [I_{ws}] \;\right]WQ^T(QWQ^T)^{-1}
L_r = P_m
\eeq
where $P_m$ is the modified power.

Since $\dot{\eta}_n \in \mathcal{N}(Q)$, there exists a vector
$\nu \in \Re^{2N}$ such that \cite{tsiotras1}
\beq
\dot{\eta}_n = \mathcal{P}_N \nu
\eeq
where $\mathcal{P}_N = I_{2N} - Q^T(QQ^T)^{-1} Q $ is the
orthogonal projection on $\mathcal{N}(Q)$.
Thus we have
\beq
\left[\; 0 \; \vdots \; \Omega^T [I_{ws}] \;\right]\mathcal{P}_N
\nu = P_m
\eeq
and the minimum norm solution of this equation is
\beq
\nu =\mathcal{P}_N \left\{
\begin{array}{c} 0 \\ \left[I_{ws}\right] \Omega
\end{array} \right\}
\left(\left[\; 0 \; \vdots \; \Omega^T [I_{ws}]
\;\right]\mathcal{P}_N \left\{ \begin{array}{c} 0 \\\left[I_{ws}\right] \Omega
\end{array} \right\} \right)^{-1} P_m
\eeq
Therefore the angular velocity input for power tracking can be
chosen as
\beq
\dot{\eta}_n =\mathcal{P}_N \nu =\mathcal{P}_N \left\{
\begin{array}{c} 0 \\ \left[I_{ws}\right] \Omega
\end{array} \right\}
\left(\left[\; 0 \; \vdots \; \Omega^T [I_{ws}]
\;\right]\mathcal{P}_N \left\{ \begin{array}{c} 0 \\\left[I_{ws}\right] \Omega
\end{array} \right\} \right)^{-1} P_m
\eeq
and the combined angular velocity input for attitude and power tracking is
\beq
\dot{\eta}_d = WQ^T(QWQ^T)^{-1} L_r + \dot{\eta}_n \label{eq05}
\eeq

\subsection{Acceleration based steering law}

For more accurate control and realistic simulation, the gimbal
acceleration term in Eq.(\ref{eq03}) should be used as the actual
control input.
The goal of gimbal acceleration based steering law is to calculate the gimbal acceleration
input which makes the real gimbal rate converge to the desired
gimbal rate provided in Eq.(\ref{eq05}).

If a Lyapunov function is defined as
\beq
V_2 = \frac{1}{2} (\dot{\gamma}_d - \dot{\gamma} )^T (\dot{\gamma}_d - \dot{\gamma} )
\eeq
then,
\beq
\dot{V_2} = (\dot{\gamma}_d - \dot{\gamma} )^T (\ddot{\gamma}_d - \ddot{\gamma} )
\eeq
So the stability condition becomes
\beq
\ddot{\gamma} = k_2 (\dot{\gamma_d} - \dot{\gamma} ) +
\ddot{\gamma_d }
\eeq
where, $k_2$ is a N$\times$N positive definite matrix.
Usually, $\ddot{\gamma_d }$ can be assumed to be small and is
neglected. \cite{schaub1}

Finally, the acceleration based steering law is derived as
\beq
\left\{ \begin{array}{c} \ddot{\gamma} \\ \dot{\Omega} \end{array} \right\}
= \left[ \begin{array}{cc} k_2 & 0 \\  0 & I_N \end{array}
\right]
\left( \dot{\eta}_d - \left\{ \begin{array}{c} \dot{\gamma} \\ 0
\end{array} \right\} \right)
\eeq

\section{Further Research Plan}

Most of all, I want to finish the simulation program code using
the Matlab. For this purpose, I also need a reasonable and
realistic simulation scenario, so I will read the simulation part of Haijun's paper.
Dave Richie is also trying to compose the simulation program using
Simulink, so we will coordinate with each other in this problem.

I also have to think over the dynamics of reference (target)
frame in deriving the stability condition and the weighting matrix
for the power tracking problem.

And then, I will apply this results to the actuator failure
problem, i.e., some of gimbals are fallen into the gimbal-lock or
some of CMGs are out of order. In the conventional reaction wheel
cases, it is possible to stabilize the attitude using only two
reaction wheel, (though it is not a continuous feedback control,)
thus I will examine a possibility of the stabilization with only
two degree of freedom.

\begin{thebibliography}{99}
\bibitem{schaub1} Schaub,H., Vadali,S.R., and Junkins,J.L.,``Feedback Control Law for Variable Speed Control Moment
Gyroscopes," {\it Journal of the Astronautical Sciences,} Vol.
46, No.3, 1998, pp.307-328.
\bibitem{schaub2} Schaub,H. and Junkins,J.L.,``Singularity
Avoidance Using Null Motion and Variable-Speed Control Moment
Gyros," {\it Journal of Guidance, Control, and Dynamics,}
Vol.23, No.1, 2000, pp.11-16.
\bibitem{ford1} Ford,K.A. and Hall,C.D.,``Flexible Spacecraft
Reorientations Using Gimbaled Momentum Wheels," {\it Advances in
the Astronautical Sciences, Astrodynamics 1997}, edited by F.
Hoots, B.Kaufman, P.J.Cefola, and D.B.Spencer, Vol.97, Univelt,
San Diego, 1997, pp.1895-1914.
\bibitem{ford2} Ford,K.A. and Hall,C.D.,``Singular Direction Avoidance Steering for Control-Moment
Gyros," {\it Journal of Guidance, Control, and Dynamics,} Vol.23, No. 4, 2000, pp.648-656.
\bibitem{tsiotras1} Tsiotras,P., Shen,H. and Hall,C.,``Satellite Attitude Control and Power Tracking with Momentum Wheels," {\it Journal of Guidance, Control, and
Dynamics} (to appear).
\end{thebibliography}
\end{document}
